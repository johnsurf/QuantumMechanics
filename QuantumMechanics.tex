%   11/11/92 211121724  MEMBER NAME  REPORT1  (TEXFILES) M  TEX
%format: latex    %hyphen: English
%\documentstyle[11pt,leqno,epsfig]{note}
%\documentstyle[11pt,epsfig]{note}
%\documentstyle[11pt,epsfig,pstimes,psmath]{note}
%\documentstyle[epsf]{article}
 
%\documentclass{article}
\documentclass[epsf]{article}

\font\twelverm=cmr12
\font\twelvebf=cmbx12
\font\tenbf=cmbx10
\font\tenrm=cmr10
\font\tenit=cmti10
\font\twelveit=cmti12
\font\elevenbf=cmbx10 scaled\magstep 1
\font\elevenrm=cmr10 scaled\magstep 1
\font\elevenit=cmti10 scaled\magstep 1
\font\ninebf=cmbx9
\font\ninerm=cmr9
\font\nineit=cmti9
\font\eightbf=cmbx8
\font\eightrm=cmr8
\font\eightit=cmti8
\font\sevenrm=cmr7

%\usepackage{amsmath,amssymb,latexsym} 
\usepackage{amsmath,amssymb,latexsym,hyperref} 
\usepackage{easybmat,graphicx}
\usepackage{multirow,bigdelim}
\usepackage{caption}
\usepackage{subcaption}
%\input ../inputs/epsf.tex

\newcommand*\hexbrace[2]{%
  \underset{#2}{\underbrace{\rule{#1}{0pt}}}}
\hypersetup{
    colorlinks=true,
    linkcolor=blue,
    filecolor=magenta,      
    urlcolor=cyan,
}

%\hypersetup{
%    bookmarks=true,         % show bookmarks bar?
%    unicode=false,          % non-Latin characters in Acrobat�s bookmarks
%    pdftoolbar=true,        % show Acrobat�s toolbar?
%    pdfmenubar=true,        % show Acrobat�s menu?
%    pdffitwindow=false,     % window fit to page when opened
%    pdfstartview={FitH},    % fits the width of the page to the window
%    pdftitle={My title},    % title
%    pdfauthor={Author},     % author
%    pdfsubject={Subject},   % subject of the document
%    pdfcreator={Creator},   % creator of the document
%    pdfproducer={Producer}, % producer of the document
%    pdfkeywords={keyword1, key2, key3}, % list of keywords
%    pdfnewwindow=true,      % links in new PDF window
%    colorlinks=true,       % false: boxed links; true: colored links
%    linkcolor=blue,          % color of internal links (change box color with linkbordercolor)
%    citecolor=green,        % color of links to bibliography
%    filecolor=magenta,      % color of file links
%    urlcolor=cyan           % color of external links
%}

%
%  `pstimes.sty' sets \rm to be Times, and `psmath.sty' sets the math
%  in Times as well.
%
\textwidth = 7.0truein
\textheight = 8.5truein
\oddsidemargin=-0.1truein
\evensidemargin=-0.1truein
\topmargin=0.1truein
%\notename{UCD/IIRPA} \notenumber{92-24}

\begin{document}
%\null
%\vspace{-2cm}
%\hbox to \hsize{\hfill UCD/IIRPA~92-24}

%\begin{center}{{\twelvebf Notes on Abstract Algebra}}
%		\vglue 1.0cm
% 	       {\tenrm JOHN R.\,\,SMITH\\}
%	       {email: jrsmith@ucdavis.edu\\}
%		\vglue 0.25cm
%	        \baselineskip=13pt
%	       {\tenit Physics Department\\
%	        University of California, Davis\\}
%                 \baselineskip=12pt
%                 {\tenit Davis, CA 95616-8677, USA\\}
%		\vglue 0.5cm
%                 {\rm April 14, 2019 \\}
%\end{center}

%
%\newmathalphabet{\oldcal}
%\addtoversion{times}{\oldcal}{cmsy}{m}{n}
\newcommand{\unit}[1]{\mbox{\elevenit #1}}
\newcommand{\subs}[1]{\mbox{\scriptsize\elevenit #1}}
\newcommand{\sign}{\text{sign}}
\def\Q2{$Q^2$}
\def\sgp{$\sigma_{\gamma p}\:$}
\mathchardef\Lcur="324C
\def\crfive{\cr\noalign{\vskip 5pt}}
\def\crten{\cr\noalign{\vskip 10pt}}
\def\cofac{\hbox{cofactor}}
\def\gij{g_{ij}}
\def\gijinv{g^{ij}}
\def\gstar{\overline{g}_{\alpha\beta}}
\def\Kronij{\delta^j_i}
\def\Z{\mathbb{Z}}
\def\F{\mathbb{F}}
\def\R{\mathbb{R}}
\def\L{\mathbb{L}}
\def\C{\mathbb{C}}
\def\a{\mathfrak{a}}
\def\b{\mathfrak{b}}
\def\c{\mathfrak{c}}
\def\c{\mathfrak{c}}
\def\r{\mathfrak{r}}
\def\o{\mathfrak{0}}
\def\half{{1\over 2}}
\def\gab{g_{\alpha\beta}}
\def\gabinv{g^{\alpha\beta}}
\def\gmunu{g_{\mu\nu}}
\def\gmunuinv{g^{\mu\nu}}
\def\gmu{\gamma^\mu}
\def\gmudag{\gamma^{\mu\dag}}
\def\gnu{\gamma^\nu}
\def\gmuc{\gamma_\mu}
\def\gnuc{\gamma_\nu}
\def\gzero{\gamma^0}
\def\gone{\gamma^1}
\def\gtwo{\gamma^2}
\def\gthree{\gamma^3}
\def\gfive{\gamma_5}
\def\gfive{\gamma_5}
\def\gzeroc{\gamma_0}
\def\gonec{\gamma_1}
\def\gtwoc{\gamma_2}
\def\gthreec{\gamma_3}
\def\qsq{{Q^2\over4E_1^2}}
\def\qr2{{Q^2\over2}}
\def\q2{{Q^2}}
\def\hy{\hat{y}}
\def\be{\begin{equation}}
\def\ee{\end{equation}}
\def\bearray{\begin{eqnarray*}}
\def\eearray{\end{eqnarray*}}
\def\Var{\hbox{Var}}
\def\char{\hbox{char}}

\def\mslim{\qopname\relax m{l.i.m}}

\newcommand{\bra}[1]{\langle #1|}
\newcommand{\ket}[1]{|#1\rangle}
\newcommand{\braket}[2]{\langle #1|#2\rangle}
\newcommand{\Tr}{{\rm Tr}}
\newcommand{\goes}{\rightarrow}
\newcommand{\define}{\overset{\Delta}{=} }
\newcommand{\ninfty}{n\rightarrow\infty}
\newcommand{\Rwp}{$\overset{\hbox{R}_{wp1}}{\sim}$}
\newcommand{\Rms}{$\overset{\hbox{R}_{ms}}{\sim}$}

\newtheorem{theorem}{Theorem}[section]
\newtheorem{lemma}[theorem]{Lemma}
\newtheorem{proposition}[theorem]{Proposition}
\newtheorem{corollary}[theorem]{Corollary}

\newenvironment{proof}[1][Proof]{\begin{trivlist}
\item[\hskip \labelsep {\bfseries #1}]}{\end{trivlist}}
\newenvironment{definition}[1][Definition]{\begin{trivlist}
\item[\hskip \labelsep {\bfseries #1}]}{\end{trivlist}}
\newenvironment{example}[1][Example]{\begin{trivlist}
\item[\hskip \labelsep {\bfseries #1}]}{\end{trivlist}}
\newenvironment{remark}[1][Remark]{\begin{trivlist}
\item[\hskip \labelsep {\bfseries #1}]}{\end{trivlist}}

\newcommand{\qed}{\nobreak \ifvmode \relax \else
      \ifdim\lastskip<1.5em \hskip-\lastskip
      \hskip1.5em plus0em minus0.5em \fi \nobreak
      \vrule height0.65em width0.4em depth0.20em\fi}
      %\vrule height0.75em width0.5em depth0.25em\fi}
      
\newcommand\xqed[1]{%
  \leavevmode\unskip\penalty9999 \hbox{}\nobreak\hfill
  \quad\hbox{#1}}
\newcommand\demo{\xqed{$\triangle$}}      
 


%\maketitle

%some definitions
%\setlength{\oddsidemargin}{0cm}
%\setlength{\textwidth}{17cm}
%\setlength{\topmargin}{-3cm}
%\setlength{\textheight}{25cm}
%\setlength{\unitlength}{1mm}
%\hsize=6.0 true in
%\vsize=8.40 true in
%\pagestyle{plain}

\parindent = 0.cm
%\vglue 0.3cm
%\rightskip=3pc
%\leftskip=3pc
\twelverm
\baselineskip=14pt
\noindent

%\input /Users/johnrsmith/tex/inputs/phyzzx.tex
%\input /Users/johnrsmith/tex/inputs/myphyx.tex
%$$p(x,D)\psi(x) = {1\over(2\pi)^n}\int_{R^n}\int_{K^n} e^{i k\cdot x}p(x,k)e^{-i k\cdot y}\,d^nk\psi(y)\,d^ny,\eqno(8)$$
%\begin{equation}p(x,D)\psi(x) = {1\over(2\pi)^n}\int_{R^n}\int_{K^n} e^{i k\cdot x}p(x,k)e^{-i k\cdot y}\,d^nk\psi(y)\,d^ny,\label{eqn:symbol}\end{equation}
%Eq.[\ref{eqn:symbol}]
%\begin{equation*}p(x,D)\psi(x) = {1\over(2\pi)^n}\int_{R^n}\int_{K^n} e^{i k\cdot x}p(x,k)e^{-i k\cdot y}\,d^nk\psi(y)\,d^ny,\end{equation*}

%\input Introduction.tex

%\input{Logic.tex}
%\input{EquivalenceRelations.tex}
%\input{SubGroups.tex}
%\input{FermatLittleTheorem.tex}
%\input{ChineseRemainder.tex}
%\input{Fields.tex}
%\input{FiniteFields.tex}
%\input{Descartes.tex}
%\input{Determinants.tex}
%\input{Vandermonde.tex}
%\input{Sylvester.tex}
%\input{Geometry.tex}
%\input{ApplicationsInMechanics.tex}
%\input{Algorithm.tex}
%\input{FourOutbounds.tex}
\section{The Operator Approach to the Harmonic Oscillator}
The Hamiltonian for the one-dimensional harmonic oscillator is given by
\begin{equation}
H = {p^2\over2m} + \half k q^2
\end{equation}

The quantum mechanical commutation relations for the operators $p$ and $q$ is 
\begin{equation}[q,p] = i\hbar
\end{equation}

Define two operators $a$ and $a^\dagger$ as follows:

\begin{eqnarray*}
a &=& \sqrt{m\omega/2\hbar}q + ip/\sqrt{2m\hbar\omega}\\
a^\dagger &=& \sqrt{m\omega/2\hbar}q - ip/\sqrt{2m\hbar\omega}\\
\end{eqnarray*}
where $\omega = \sqrt{k/m}$ is the classical angular frequency of the harmonic oscillator.\\

Form the product
\begin{eqnarray*}
 \hbar\omega a^\dagger a &=& \hbar\omega \bigg( \sqrt{ {m\omega\over 2\hbar}} q  - {ip\over\sqrt{2m\hbar\omega}}\bigg) \bigg( \sqrt{ {m\omega\over 2\hbar}} q  + {ip\over\sqrt{2m\hbar\omega}}\bigg)\\
  &=&  {p^2\over2m} + \half k q^2  +{i\omega\over2}(qp - pq)\\
  &=& H - \half \hbar\omega 
 \end{eqnarray*}
 
 Using the above we can rewrite the Hamiltonian for the one dimensional harmonic oscillator as
 
 \[H = \hbar\omega(a^\dagger a + \half)\]

\clearpage

%\input{Appendices.tex}
%\input{Acknowledgements.tex}
%\clearpage

%\begin{thebibliography}{9}
%\bibitem{Schreier and Sperner} {O. Schreier and F. Sperner,``Introduction to Modern Algebra and Matrix Theory'', Second Edition, Chelsea Publishing, (1959).}
% \bibitem{Protter}{M.H. Protter and C.B. Morrey, ``A First Course in Real Analysis'', Springer-Verlag, Inc., New York (1977).}
% \bibitem{Hall}{F. M. Hall, ``An Introduction to Abstract Algebra'', Volume 2, Cambridge University Press, (1969).}
% \bibitem{Stensby}{John Stensby, ``EE603 Class Notes'', \url{http://www.ece.uah.edu/courses/ee385/}, University of Alabama, Huntsville (2016).}
% \bibitem{Baudoin}{Fabrice Baudoin,``Stochastic Processes and Brownian Motion'',\url{http://www.math.purdue.edu/~fbaudoin/MA539.pdf}, Purdue University, Indiana (2010).}
% \bibitem{vanZanten}{Harry van Zanten,``An Introduction to Stochastic Processes in Continuous Time'', \url{http://www.math.vu.nl/sto/onderwijs/sp/sp_2007.pdf}, University of Leiden (2007).}
% \bibitem{Weiner1}{Howard Weiner, UC Davis Mathematics Department, private communication (2007-2017).}
% \bibitem{Mahalanobis}{Mahalanobis, Prasanta Chandra (1936).``On the generalised distance in statistics'', Proceedings of the National Institute of Sciences of India. 2 (1): 49�55.}
% \bibitem{Einstein}{Albert Einstein, ``Investigations on the Theory of the Brownian Movement'', Dover Publications, Inc.(1956).}
% \bibitem {Ross1} {Sheldon Ross, ``A First Course in Probability'', First Edition, Macmillan Publishers Co, Inc., New York, NY (1976).}
% \bibitem {Ross2} {Sheldon Ross, ``Introduction to Probability Models'', Sixth Edition, Academic Press, San Diego, CA  (1997).}
% \bibitem{Parzen}{Emanuel Parzen, ``Stochastic Processes'', Holden-Day, Inc., San Francisco, CA (1962).}
% \bibitem{Medhi}{Jyotiprasad Medhi, ``Stochastic Processes'', Wiley Eastern Limited, New Delhi (1983).} 
% \bibitem{Jazwinski}{Andrew H. Jazwinski,``Stochastic Processes and Filtering Theory'', Dover Publications, Inc., Mineola, NY (2007).}
% \bibitem{Wiener1}{Norbert Wiener, ``Cybernetics: or Control and Communication in the Animal and the Machine'', The MIT Press, Cambridge, MA (1948).}
% \bibitem{Wiener2}{Norbert Wiener,``Extrapolation, Interpolation, and Smoothing of Stationary Time Series with Engineering Applications'', First Paperback Edition, The MIT Press, Cambridge, MA (1964). }  
% \bibitem{Wiener3}{Norbert Wiener, ``Generalized Harmonic Analysis'', Acta Mathematica, V. 55, p. 117 (1930).}  
% \bibitem{Oskendal1}{Bernt  {\O}skendal, ``Stochastic Differential Equations, An Introduction with Applications'', Third Edition, Springer-Verlag, Berlin (1992).}
% \bibitem{Oskendal2}{Bernt  {\O}skendal, ``Stochastic Differential Equations, An Introduction with Applications'', Sixth Edition, Springer-Verlag, Berlin (2005).}
% \bibitem{Doob}{J. L. Doob, ``Stochastic Processes'', John Wiley \& Sons, Inc., New York, (1953).}
% \bibitem{Sjogren}{Lennart Sjogren. ``Brownian Motion: Langevin Equation'', Chapter 6, \url{http://physics.gu.se/~frtbm/joomla/media/mydocs/LennartSjogren/kap6.pdf}}
% \bibitem{CalTech}{``Statistical Mechanics'', Physics 127b \url{http://www.cmp.caltech.edu/~mcc/Ph127/b/Lecture16.pdf}}
%\end{thebibliography}

%\begin{description}
%\elevenitem[\elevenit Reflexive.] Since $x = e^{-1}xe$, x is conjugate to itself.
%\elevenitem[\elevenit Symmetric.] If $y=g^{-1}xg$ we have $x=gyg^{-1} = g^{-1})^{-1}y g^{-1}$.
%\elevenitem[\elevenit Transitive.] If $y=g^{-1}xg$ and $z=h^{-1}yh$ we have $z=h^{-1}g^{-1}xgh=(gh)^{-1}x(gh).$
%\end{description}

%\begin{eqnarray*}
%   P_1 & = & (E_1,0,0,-E_1),      \\
%   P_2 & = & (E_2,E_2\sin\theta\cos\phi,E_2\sin\theta\sin\phi,
%               -E_2\cos\theta),   \\
%   P_3 & = & (E_3,0,0,\beta E_3),      \\
%   q   & = & (E_1 - E_2,-E_2\sin\theta\cos\phi,-E_2\sin\theta\sin\phi,
%                    E_2\cos\theta - E_1).
%  \end{eqnarray*}

%\begin{figure*}[hb]
%\epsfysize=6cm %%%% whatever vertical size you want in cm or inches
%\centerline{\epsfbox{user$1:[smith.tex.eps]feyn.eps}}
%\caption[fig1]{Definition of Kinematic Variables.}
%\label{fig:feyn}
%\end{figure*}

%  \begin{eqnarray*}
%   y\     & \approx & 1 - {E_2\over E_1} {(1+\cos\theta)\over2}, \\
%    \qsq \ & \approx & {E_2\over E_1}{(1-\cos\theta)\over2}, \\
%    x      & \approx & {Q^2\over sy}.
%  \end{eqnarray*}

%\[ \begin{array}{lll} 
%  {\cal M}^{++}=({\cal M}^{22} + {\cal M}^{11})/2, 
%& {\cal M}^{+0}=({\cal M}^{23} -i{\cal M}^{13})/\sqrt{2},
%& {\cal M}^{+-}=({\cal M}^{22} - {\cal M}^{11})/2, \\ [8pt]
%  {\cal M}^{0+}=({\cal M}^{32} +i{\cal M}^{31})/\sqrt{2}, 
%& {\cal M}^{00}= {\cal M}^{33}, 
%& {\cal M}^{0-}=({\cal M}^{32} -i{\cal M}^{31})/\sqrt{2}, \\ [8pt]
%  {\cal M}^{-+}=({\cal M}^{22} - {\cal M}^{11})/2, 
%& {\cal M}^{-0}=({\cal M}^{23} +i{\cal M}^{13})/\sqrt{2}, 
%& {\cal M}^{--}=({\cal M}^{22} + {\cal M}^{11})/2. \\ [8pt]
%\end{array} \] 

%\[ \begin{array}{lll} `
%  {\cal M}^{++}=({\cal M}^{22} + {\cal M}^{11})/2, 
%& {\cal M}^{+0}= {\cal M}^{23}/\sqrt{2},
%& {\cal M}^{+-}=({\cal M}^{22} - {\cal M}^{11})/2, \\ [8pt]
%  {\cal M}^{0+}= {\cal M}^{32}/\sqrt{2}, 
%& {\cal M}^{00}= {\cal M}^{33}, 
%& {\cal M}^{0-}= {\cal M}^{32}/\sqrt{2}, \\ [8pt]
%  {\cal M}^{-+}=({\cal M}^{22} - {\cal M}^{11})/2, 
%& {\cal M}^{-0}= {\cal M}^{23}/\sqrt{2}, 
%& {\cal M}^{--}=({\cal M}^{22} + {\cal M}^{11})/2. 
%\end{array} \]
%
%\begin{thebibliography}{99}
%\bibitem{BCDMS} J.J.Aubert et al., EMC Collaboration,
% Nucl.Phys. B259(1985)189;\\
% A.C.Benvenuti et al., BCDMS Collaboration, Phys.Lett.B223(1989)485``
%\bibitem{kaiserf}     F. Eisele, {\em First Results from the H1
%Experiment at HERA}, and
%                      F. Brasse, {\em The H1 Detector at HERA},
%                      Invited talks, Proceedings of the
%            26th International Conference on High Energy Physics,
%                   Dallas (1992) and DESY preprint 92-140 (1992)
%\bibitem{H1-249} I.\, Abt, J.R.\,Smith, {\em MC Upgrades to Study
%Untagged Events}, H1 Internal Note H1-249, (1992).
%H1-249 considered only the diagonal elements of the Photon Flux
%for photons coupled to Spin-1/2 particles.
%\end{thebibliography}
 %
 \end{document}
